% !TEX program = xelatexmk
\documentclass[10pt, b6paper, openany]{memoir}
\usepackage{gtszrcp}

\usepackage[
backend=biber,
style=authoryear,
citestyle=authoryear
]{biblatex}
\addbibresource{lic-ko.bib}

\usepackage{url}

\usepackage{xltxtra}

\setmainhangulfont{NotoSerifCJK-Medium}[
Extension=.ttc,
UprightFont=NotoSerifCJK-Light,
BoldFont=NotoSerifCJK-Bold,
AutoFakeSlant=0.2
]
\setcounter{secnumdepth}{0}

% Title, authors, date.
\newcommand{\booktitle}{틈새의 삶}
\newcommand{\translator}{서한교}
\newcommand{\subtitle}{희망 없이 죽어가는 행성에서 공허로부터 의미를 창조하는 노바토레주의자의 에세이}
\newcommand{\fulltitle}{\booktitle{}~--- \subtitle{}}

\title{\booktitle{}}
\author{프리드리히 루럴 루시퍼} 
\date{2018}
\posttitle{ --- \subtitle{}\end{textbf},}
\postauthor{, (\translator{} 옮김), }

\begin{document}

\frontmatter
\begin{titlingpage}
\begin{flushleft}
\maketitle
\end{flushleft}
\end{titlingpage}
\tableofcontents

\mainmatter
\begin{article}
\chapter[틈새의 삶]{\fulltitle}
\section{}
솔직히 말해보자. 

현재의 정치-경제적 기반은 인위적 기후변화에 대한 중지, 반전, 심지어 획기적인 ``피해 통제''의 가능성을 제공하지 않는다. 시장과 법률 제도는 대기 중 350ppm의 탄소와 야생의 자연, 살기 좋은 생태 환경, 그리고 모든 동물, 인간과 비인간들의 자유에 정반대되는 것에 관심이 있다. 이전에는 상상할 수 없었던 규모와 힘, 기능을 가진 군사--감옥 복합체에 의해 그러한 이해관계가 강제되고 유지된다. 더 나아가 현재의 생태학적 위기가 해소되지 않는 한 이전에 구상된 ``혁명''은 커녕 그 비슷한 것이라도 발생할 일이 없을 것이다. ``혁명적 예외''가 일어날 수도 있지만, 완전히 미쳐버리지 않는 한 어떠한 진지한 희망도 갖긴 힘들 것이다.

이 모든 것은 망할 생태학적 재앙; 기하급수적인 종의 멸종, 증가하는 삼림 벌채, 비안정적이고 유한한 에너지원, 인구 증가와 한정된 자원으로 인한 부담, 확산되는 해양의 죽음의 구역, 이에 따른 어업 붕괴, 이전에는 상상할 수 없었던 표토 감소율, 그리고 지속적이고 심각한 가뭄이 더 큰 빈도로 발생하는 문제에 대한 논의는 배제하는 것이다. 

나는 우리가 수사적으로 솔직할 수 있는지 묻는 것이 아니다. 나는 진실하게 그리고 실질적으로 묻고 있다. 만약 당신이 이러한 개념을 사실로 혹은 `기본적으로는 사실'이라고 받아들인다면, 계속 할 수 있겠는가? 당신은 무관심한 채로 있을 것인가? 헤로인을 흡입하거나, 자살할 것인가? `존엄하게' 숲 공동체에 합류할 것인가? 세계여행을 하며 미래 사회를 위한 디지털 타임캡슐에 저장할 사진을 찍거나 기사나 쓸 것인가? 정유공장, 발전소, 생명공학단지에 불을 지르고 수감생활을 받아들이겠는가? 당신은 헛된 입법 시도와 에어컨을 튼 쇼핑몰에 쳐박혀 있을 것인가? 그냥 맛있는 식사를 하고 하루를 즐기겠는가?

\emph{하루를 즐길 것인가?}

\section{}

나는 내가 무엇을 해야할 지 모르기 때문에 이 글을 쓰고 있다. 이 에세이는 비슷한 이해를 공유하는 다른 사람들을 찾으려는 희망에서 비롯된, 이런 터무니없는 곤경에 직면하여 어떤 의미를 만들어내려는 나의 시도다. 거짓을 진실이라 믿는 문화에서 정직하다는 것은 고독의 사막으로 들어가는 것이다. 내가 사막에 들어가야 할 때, 유목민 무리들과 함께 한다면, 나의 생존 가능성이 훨씬 더 높아질 것이다. 홀로 들어간다면 나는 죽을 것이다. 만약 그렇게 된다면 이 에세이는 일종의 허무주의적 개인 광고로 읽힐 것이다.

위에서 내가 표현했던 견해는 2011년 익명의 저자에 의해 아나키스트 도서관 웹사이트\footnote{https://theanarchistlibrary.org/}에 공개된 책자 \booktitlewrap{사막}\parencite{bk:desert2011}에서 아름답고 분명하게 표현되었다. 그 책자는 본질적으로 우리의 곤경에 대한 사실적 근거를 제시하고, 일부 사람들이 적절히 ``녹색 허무주의(Green Nihilism)''라고 불러온 사고방식을 형성하는 급진적 환경운동, 그들의 분석, 그리고 그들의 가정(假定)을 비판한다. 나는 \booktitlewrap{사막}의 관점에 기초해 이 글을 이어갈 것이다. 이 에세이를 쓰려는 나의 충동을 더 잘 이해하기 위해서, 나는 \booktitlewrap{사막}부터 읽을 것을 권고한다.

\section{}

우리는 유령의 틈새에 살 수 있을까? 

\booktitlewrap{사막}에 따르면, ``지구적 미래는 없다.'' 지배 공간과 자유 공간은 항상 존재할 것이다. 우리는 비관적으로 보일지도 모르는 모든 사회를 결코 해방시킬 수는 없지만, 그들 역시 사회 전체를 완전히 통제할 수는 없다. 어떤 때, 어떤 곳이든 지배적인 문화와 그 부역자들이 보거나 만질 수 없는 공간에 사람과 생물이 모일 수도 있다. 이 공간에서 우리는 우리가 원하는 어떤 규칙으로 자유롭게 놀 수 있다. 이 이외의 장소는 완벽한 통제와 정책으로 결코 감시(Eye)를 피할 수 없어 보인다. 이 후자의 범주에 속하는 공간은 당분간 죽은 공간으로 내버려두어야 할 수도 있다. 

지배 체제의 논리는 본질적으로 장엄하다고들 믿는다. 그저 총, 처벌, 교도소에 의해 유지되는 믿음일 뿐인데 말이다. 그러나 이러한 믿음의 집행자가 없는 곳에도 그러한 믿음이 존재할 것인가? 그 밖으로 한 발짝 밖에서도 그 믿음이 유효할 수 있을까? 그들의 장엄한 붕괴가 가능한 시공이 존재할까? 내가 이해하기로, 그 대답은 `그렇다'이다. 이 긍정은 더 큰 위기 앞에서도 의미를 가질 수 있다. `그렇다'. 그것은 이러한 붕괴를 충분히 활용할 수 있을 만큼 유연하고 혁신적이며 탄력적인 이들이 체계의 약점을 활용하기 위한 소정의 낙관에 기초를 제공할 수 있을 것이다. 

이탈리아의 개인주의적 아나키스트인 렌조 노바토레(Renzo Novatore)는 20세기 초 발표한 \articletitlewrap{창조적인 무를 향하여(Toward the Creative Nothing)}\parencite{ibk:renzo2012}라는 제목의 에세이에서 다음과 같이 말한다.

\begin{quote}
아나키즘은 역사의 무대에서 서로를 따르는 모든 사회에 대항하는 소수의 귀족적 아웃사이더들의 영원한 투쟁이다.
\end{quote}

노바토레가 아나키즘을 귀족적이라고 묘사한 것은 노바토레의 특유의 인습 타파적 과장법이다. 그는 그와 격렬히 논쟁하던 정체된 공산주의적 아나키스트들과 그들의 조직주의적 교리를 폭로하기 위해 저런 식으로 표현했다. 그리고 그가 떠난 지금에도 그 과장법이 유효하다는 점에는 의심의 여지가 없다 – 오늘날까지도 엄격한 사회주의-아나키스트들을 분노하게 할 수 있으니까 말이다. 노바토레의 `귀족주의'는 예수의 `만왕萬王의 왕'이라는 지위가 근본적으로 반권위주의적인 방식으로 `왕'이라는 사회적 범주를 불안정하게 만드는 역할을 했던 것과 거의 같은 역할을 한다. 자신의 자유를 얻은 이러한 자율적 개인들은 확실히 소수를 이루지만, 부르주아 계급과는 달리, 이 소수는 필연적으로 수단과 방법을 가리지 않고 스스로를 확장하려고 한다. 이러한 아나키즘은 판옵티콘(Panopticon)이 볼 수 없는 지구상의 그러한 지역에서 번성하고 있다. 저 어두운 골목길과 흙길이 이 귀족들의 시작 지점이다.

국가에 내제된 붕괴의 가능성을 포착하고 타격하는 일은 엄밀히 말해서 물리적인 일이라고만 할 수 없다. 그것 역시 철학적인 것이다. 국가와 지배체제의 패권은 우리 각자가 계승하고 정직의 중요성에 근거하여 의문을 제기해야 하는 허위와 거짓된 가정으로 이루어진다. 무신론자들은 이러한 과정을 상당히 잘 개척해 왔다. 그들이 유신론자들에게 요구하는 것은 간단하다. 아무리 무모하더라도 얼마든지 주장하라. 네가 증거를 제시한다면 나는 그것을 믿을 것이다. 

무신론자들이 어떤 주장을 하든지 그 입증의 책임이 청구인에게 있다는 점을 증명하기 위해 자주 사용하는 고전적인 비유가 있다. 만약 어떤 사람이 금성의 궤도에 찻주전자가 있다고 주장한다면, 그 찻주전자는 너무 작아서 어떤 망원경으로도 볼 수 없다고 주장한다면 나는 아무런 증거도 없이 그들의 주장을 전적으로 믿어야 하는가? 물론 그렇지 않다. 또한 나는 찻주전자의 존재 가능성이, 존재하지 않을 가능성과 같다는 불가지론자의 입장을 믿지 않는다. 어떤 근거나 암시적인 증거가 나올 때까지 내가 그것을 기본적으로 거짓이라고 가정하는 것은 터무니없는 주장이다. 그리고 신에 대한 믿음도 다르지 않다. 

그러나 무신론자들은 대개 이런 사고방식을 취하지 않았다. 어떤 종류든지 본질적 또는 객관적 의미에 대한 주장은 유사하게 근거 없는 형이상학적 주장 위에 놓여 있다. 의미가 있는 모든 것은 우리, 주체가 그렇게 만들었기 때문에 그런 것이다. 문제는 우리의 주관적인 의미 결정 과정이 정말로 우리 자신의 것인지에 있다. 객관적 의미에 대한 요구는 개인과 그들이 만들어내는 의미 영향을 미치는데, 이들이 만들어내는 의미는 대체로 기생적이고, 외부 사회 질서, 계급 혹은 이념의 영향력을 증대하는데 복무한다. 이러한 주장들은 그 주체를 그 매개체의 노예나 포로가 되게 한다. 그러나 이러한 굴종은 주체가 객관적, 즉 도덕적, 실존적, 그리고 정치적 의미가 있다고 주장하는 사람들에게 `그래서 증거는 어디에 있는가?'라는 단순한 질문을 던질 때 비로소 끝난다.

그들은 이러한 질문에 대해 짜증을 내거나 공허하고 자기 과시적 장광설만을 보여줄 것이다. 그들은 원점으로 돌아오거나 세속적인 모습으로 `공익'과 `혁신', 혹은 신에 대한 논쟁을 할 수도 있다. 허무주의적 아나키스트는 이것을 권력자들이 자신의 신분과 자리를 그대로 유지하려는 필사적이고 가련한 시도로 본다. 

\section{}

그래서 우리는 변증법을 살해했다. 좋다! 허무주의적인 배경에서, 본질에 대한 주장들을 거부하면서, 내가 무로부터 나만의 의미를 자유롭게 창조할 수 있는 곳은 이곳이다. 노바토레가 말했듯이, 나는 창조적인 무를 향해 걸어간다. 아나키스트 귀족들의 자기 확장 과정은 내 마음속에서 가장 높은 형태의 사랑이며, 나에게 있어서 가장 큰 본질의 원천이다. 사회의 가장 어두운 구석에 있는 악당들의 위치로부터, 그는 손짓하며 말한다. `자, 우리가 다른 사람들을 일원으로 맞아들일 수 있도록, 나와 함께 가장 높은 수준으로 자기실현을 하고, 우리의 삶을 이런 비참한 체제에서 건져내고, 안개 자욱한 밤에 그들을 모조리 쓸어내자!'. 

내 죽음이 어렴풋이 보인다! 어느 날 갑자기 올 수도 있고, 이 죽어가는 곳에 수십 년을 더 보낸 후에 올 수도 있다. 내가 가장 높은 상태에 도달하기 위해 기다리는 것은 내게 어떤 도움이 될까? 무엇을 기다리는가? 유일하게 확실한 것은 현재 나 자신의 경험에 있다! 그리고 내 주위의 다른 사람들에 대해서도 마찬가지로 허위인가? 같이 뛰어들면 안 될까? 확실히, 일, 임대료, 배고픔, 지루함, 무지와 계승된 도덕과 이상, 경찰, (급진주의자와 개혁주의자 할 것 없이) 이데올로기로부터 비롯된 숨막히는 도덕적 고집, 그리고 여전한 다른 사람들의 고통. 이러한 자기실현에 대한 족쇄는 나와 내 주변 사람들에게 공통적으로 존재한다. 깨져야 할 족쇄가 많다. 백지장도 맞들면 낫다는 '노동 윤리`의 선구자들의 옛 금언을 떠올려야 한다.

1916년 렌조 노바토레의 동시대 사람인 브루노 필리피(Bruno Filippi)는 다음과 같이 말했다.

\begin{quote}
오늘 저녁, 여느 때처럼 책을 읽다가 한 구절이 선명하게 떠올라 숙고하기 위해 독서를 중단했다. 나는 그때 마침 사색에 잠겨 있었는데, 내가 보고 있던 방 여기저기로 멍하니 눈을 돌렸고, 게다가 침대 위에 앉아 있는 내 자신을 보았다. 내가 아니었지만, 여전히 나였다. 그는 완전히 나와 같았기 때문이었다. 나는 놀라서 말없이 바라만 보았고, 그, 또 다른 나 역시 나를 바라보았지만, 아이러니한 미소를 짓고 있었다. 

`누구요?' 나는 그에게 물었다. `당신의 그림자.' 그가 대답했다. `무얼 좀 의논하러 왔소.' `그렇다면 의논해 봅시다.' 내가 대답했다. 

`음. 너는 왜 아나키스트인가?' `왜냐고? 현재 우리는 지배자들에게 이용당하고 짓밟히고 있기 때문이지.' 

``아주 입에 발린 헛소리군, 들어봐! 넌 아나키스트이고, 니가 왜 아나키스트인지는 너도 몰라. 모든 사회에는 결국 화형에 처해지거나, 십자가에 매달리는 혁신자들이 있어. 나는 이것을 항상 알고 있었어. 그렇게 그 모든 꿈을 가진 이런 혁신자들은 비참하게 실패했고 희생되었지. 그것이 어떤 것이든 어떤 개인이 품었던 어떤 쇄신도 그 사람이 죽은 지 오랜 시간이 지난 후에 야 일어나기 때문이야. 그리고 이것이 너희 아나키스트들에게 일어날 일이다. 너는 너의 이상 중 그 어떤 것도 이루어지는 것을 보지 못한 채 죽게 될거야. 아나키즘 사회에 살고 있을지도 모르는 네 다음 세대는 더 높은 이상을 갈망할테지만 그 또한 아무것도 이루지 못한 채 죽을 것이다. 악순환이지, 영원히 다람쥐 쳇바퀴 도는 것이지.''\parencite{ibk:bruno2008}
\end{quote}

그만! 내가 관심 있는 건 널 설득하는 게 아니야. 그건 버리고 망할 네 논설이나 펴라고. 나는 이것을 나 자신과 그것에 공명하는 사람들을 위해 쓰고 있다. 그 이상은 네 혼자 힘으로 해야 해. ``합리적으로 진실을 폭로''하는 담론에 대한 나의 관심이 줄어든다. 지식인들의 끝없이 반복되는 말장난은 구역질이 날 지경이다. 세상이 불에 타는 와중에 내가 평생을 담론에 바칠 수 있을까, 그러면 죽기 전에 잠시나마 진실의 삶에 대한 이념적 몸부림 속에서 쉴 수 있을까? ``친구들아, 이제 논쟁하라. 행동의 왕국, 인생의 즐거움의 왕국이 눈앞에 왔노라!'' 나는 거절한다 -- 상아 십자고상은 삶을 부정하고 자유를 더럽히는 병든 파충류를 위한 아편굴이다. 그리고 나는 기회가 있을 때마다 그들과 맞서 싸울 것이다!

갑자기 ``세상을 구하는 것''이 죽기 전에 우리에게 선물을 주었다. 우리로 하여금 현재를 그리워할 만큼 그렇게 미래와 사회 전체를 열심히 그리지 않도록 하는 선물. 낙관주의의 비대한 시체로부터 비슷한 낙관주의가 기어오르지만, 손에는 볼트 커터를 든 완고한 눈의 난장이일 뿐이다.

그는 앞장 서서 우리를 사막으로 인도한다! 다루기 힘든 불안의 틈새 안으로! 생명을 위하여! 

\section{}

로켓은 ``인류를 구하기 위해'' --  폭도인 우리와 유일한 공통점이라곤 같은 인류라는 점 말고는 없는 -- 부르주아 계급의 발기불능 환자나 머저리같은 아이들을 가득 싣고 연료통을 가득 채워 우주를 향해 출발한다. 연료가 모두 소진되면 그들은 하나씩 로켓 밖으로 내던져질 것이다. 조종사들이 우주를 포류할 때까지 완전한 고독 속으로. 

우리의 이상도 마찬가지다! 우리는 우리가 보는 문제들에 대한 해답을 찾고, 우리의 답을 공유하는 사람들의 사회적 상황을 안다. 우리는 당에 가입하고, 채식주의자가 되고, 시위를 한다. 그러나 정직한 혁신가들은 각 상황에서 많은 모순점, 장벽의 약점 -- 바로 연료 탱크가 밑으로 떨어지며 `대답'이 대체된다는 것을 발견한다. 우리는 약점들을 밀어내고, 각각 지난번보다 작고 확실히 광적인 새로운 사회적 상황들 속에 숨겨진 입구를 찾아내고, 그 과정을 몇 번이고 반복한다. 결국 모든 이데올로기가 실행되고 폐기되고, 조종사는 완전히 순수한 고립에 내버려진다. 허무와 다시 접촉하게 된 우리는, 확신의 외로움이 서린 성취할 수 있는 자질을 구비한 채 한때는 견딜 수 없었던 오물과 어리석음이 기쁨이 된 지구의 표면으로 자유롭게 떠밀려 돌아온다. 바로 이곳에서 새로운 시시포스의 꿈이 태어난다! 

나는 허무주의에 머무른 적이 있고, 엄청나게 고통스러워하고 괴로워하며 그곳에서 지냈고, 시시포스의 꿈에 적합한 몽상가였다. 이 꿈들은 무엇인가? 우리의 무의미한 곤경 앞에서 나는 무에서 어떤 의미를 창조해 낼까? 다음은 무엇일까? 

객관화 된 구조에 대한 부정이 무엇보다도 중요하다. 일, 돈, 상품, 선입견, 도덕성 모두, 심지어 -- 아마도 특히 -- 객관화 된 구조를 부정하는 금언까지. 시계는 우리의 삶을 원자재로 바꾸고, 돈은 산송장을 춤추게 하고, 공장들은 개개의 나무를 나무 모양의 열쇠고리로 만들고, 도살장은 개성과 욕망을 가진 존재들을 구역질나는 대중들을 먹이기 위한 고기로 바꾼다. 그러한 모든 구조는 분노로 가득한 웃음 속에서 파괴되어야 한다. 우리 삶의 저열한 종속에 대한 모든 증오는 우리가 적합하다고 생각하는 뿌리로 보내져야만 한다. 

우리는 경제를 파괴할 것이다. 무급 인턴, 창녀, 채무자, 쓰레기통을 뒤지는 거지들(Dumpster Divers)의 무리가 허세를 부리며 폭력적으로 무대로 올라간다. 우리가 온건파와 기생충들을 덮칠 때, 그들의 뱃속은 뒤틀릴 것이고 도둑의 웃음소리가 모든 것에 스며들 것이다. 우리 안에 묻혀 있는 그들의 관습의 씨앗들은 우리가 흥청거리는 반항으로 비열하게 이용하고, 그들의 음식을 거리낌 없이 게걸스럽게 먹어치우고, 우리 패거리, 우리 자신들의 이름으로, 거칠고 우리의 욕구들을 충족시키는 모든 것들의 이름으로 모든 도덕성을 강제로 제거하면서 가장 부르주아적인 경향을 혐오할 때, 기대할 수 있을 것이다. 혁명에 대한 세속적 그리스도교의 모습은 죽고, 폐지되었던 폭력은 욕망을 다시 무대로 돌아오게 했다. 

사회의 유령의 주변부들은 우리가 존속하고 잘 살기엔 과도한 수단들 제공한다. 쓰레기통에는 우리가 보존할 수 있는 것보다 더 많은 음식들이 쌓여있다. 고속도로에는 차량과 트럭에 의해 비극적으로 살해된 사슴의 시체가 줄지어 있지만, 우리는 그 고기를 저장함으로써 그것을 최대한 활용할 수 있다. 부두의 화물은 방치되어 있고, 우린 기꺼이 그것들을 차지할 수도 있다. 대부분의 화물 열차와 대부분의 고속도로 여행객들의 차, 생각할 수 있는 곳엔 공짜 자리가 있다. 시골의 모든 낙후지역과 쇠퇴한 죽은 도시엔 잘 이용되기를 기다리는 버려진 집들이 있다. 사회 복지 제도는 우릴 진정시키기 위해 여전히 우리의 분노를 진지하게 받아들이고 있다. 은행 경비원들은 마리화나 연기 속에 나자빠져있다. 이 모든 것들이 바로 우리가 누워 허기를 달래고 몸을 덥히며 더 나은, 더 환상적인 내일을 위한 계획을 세우기 위한 것들이다.

이 모든 것 또한 노동 거부의 역할을 한다. 노동은 생명의 흡혈귀이고, 시간의 객관화이며, 욕망의 십자가 형벌이다. 아나키즘 노동주의자(workerist)의 논증과 계율(fatwahs)의 보수적인 측면 중 하나는 그들 자신도 우리 주변에서 ``직업을 가지라''고 간청하는 반동분자들과 전혀 다르지 않다는 것이다. 그러나 태곳적부터 사기꾼들을 먹여 살린 것은 하층민들이다. 우리는 나무를 쪼개고, 떨어진 콩을 줍고, 가방 무게를 재며, 고철을 줍고, 엿 같은 일을 하며 그곳에 있다! 우리는 DVD를 훔치고, 동전 통을 흔들며, 귀금속을 떼다 팔고, 떨을 팔고, 중고장터에서 오토바이를 고쳐 되팔고, 식권을 현찰로 교환하고, 외벽 청소를 하고(나중에 그 집을 털고), 폭발물, 밀주, 소총, 그리고 마리화나와 개비담배를 팔며 그곳에 있다. 우리 조직과 패거리가 사기의 기술을 완벽하게 갖추도록 하라! 우리는 추접하고 거칠고, 우리는 세련되고 예리하며, 우리는 민첩하고 가석방 감독 기간은 끝났다. 솔직히 말해서, 우리는 정장 빼입고 엉덩이에 힘 꽉 주며 사는 것엔 관심이 없다.

관습적 업무의 필요성에서 벗어난, `귀족적 아나키스트 패거리의 확장'이 그 어느 때보다 실현 가능하게 되었다. 나는 사회와 침체와 함께 내가 존경하는 투쟁적 자유 개인들 사이에 나의 무리를 집결시키고, 모였다가 갈라지고, 동의하지 않고 파괴하는 괴팍하고 야성적인 동료들을 모으려고 한다. 우리는 숲, 산, 조차장, 그리고 빈민가로부터, 자유를 속박당한 감옥의 더 깊은 곳에 수감된 사람들에게 시선을 돌리고 구체적 사실이나, 그 필요성에 상관없이 그들의 삶, 행동, 그리고 욕망으로 변화되는 과정에서 자라나는 환상으로 만들어진 벽을 허물기 위한 음모를 꾸미고 그것을 잘 유지시킬 수 있다. 조만간, 우리는 가족, 가축화로부터 회복 중인 비인간들, 다른 능력을 가진 사람들, 그리고 우리 중 남들보다 덜 의욕적인 모든 사람들을 지원하는 데 필요한 규모의 경제에 이를 수 있을 것이다. 

하지만 이것만으로는 충분하지 않다! 단순히 전초기지, 행운이 깃들 수 있는 공간을 세우는 것은 그 자체로는 자위적인 노력이다. 대신, 유목민 사회기반시설과 강제 수용된 물품에 대한 반란군들의 보급소 창설이 훨씬 더 큰 투쟁을 위한 중추 역할을 해야 한다. 이것들은 우리가 꿈꾸고 계획하는 보금자리, 우리가 사랑과 동지애를 나누는 오두막들이다. 이곳, 이 틈새에서 단순히 우리의 삶을 영위하는 것만으로 충분한가? 

\begin{quote}
젊었을 때 그(노바토레)는 아나키스트들의 아콜라(Arcola) 그룹에 가입했지만, 그들이 그토록 간절히 기다리고 있는 새로운 사회의 조화로움과 제한적인 자유에 만족하지 못했다. 그는 ``나는 기존 사회의 압제를 파괴하는 데 있어 당신과 함께 하지만, 당신이 이 일을 완수하고 새롭게 건설하기 시작했을 때, 그때 나는 당신을 반대하며 당신을 넘어서 가겠다.''라고 말했다.\parencite{atc:enzo1950}
\end{quote}

그 모든 내용들이 그들의 공동체에 단순히 머물러 있기 위해서는, 당신이 내 앞을 가로막는다면, 야생의 독수리처럼 체제의 부패하고 있는 내면을, 그리고 당신을 공격하면서, 내가 당신을 넘어서 갈 것임을 분명히 해야 한다. 우리의 천막(Tippies)과 무단 점거한 집에서 나는 힘을 얻을 것이다. 나는 이 사회에 대한 나의 증오를 독한 술로 발효시킬 것이다. 행동 없이 이 술을 어떻게 정제할 수 있을 것인가? 

유목민적 삶의 체제에 틈새가 있는 것은 확실하지만, 공격할 기회를 제공하는 현 체제에도 틈새가 있다. 학대받은 아이가 한밤의 숲으로 도망가기 전에 주저없이 주정뱅이 의붓 애비의 고환으로 걷어 차듯 우리는 훈련을 받은 것처럼 싸워야 한다! 투쟁의 모습이 어떠해야 하는지에 대한 고전적이고 영웅적인 관념에 얽매이지 마라. 우리도 압제자들처럼 더러운 싸움을 해야 한다. 

죄수들이 있는 곳마다, 탈출과 반란의 가장자리마다 저항자들이 있다. 이들이 바로 우리가 도와야 할 개인들이다. 나는 학교에서 두들겨 맞은 아이들의 얼굴을 보았고, 그들이 감금되어 있다는 사실에 대한 혐오감으로 내 속이 뒤집힌다. 내가 당신을 위해 무엇을 할 수 있을까? 도살장은 이익을 위해 밤낮으로 짐승의 시체를 던져놓고, 그 시체에 무자비한 칼날을 들이댄다. 어떻게 하면 내가 당신의 고통을 끝낼 수 있을까? 자유에 대한 죄목으로 내 친구들은 국가에 의해 수감되었다. 감옥의 가장 약한 지점은 어디인가? 두목들, 강간범들, 경찰들, 학대자들, 마약 거래상들, 공해기업가들의 집 주소는 어디인가?

생각해보자. 만약 내가 내일 밤 총으로 무고한 사람들을 대량 살상할 계획을 세우고 있는 사람을 알고 있다면, 그리고 그가 총을 어디에 보관하고 있는지 안다면, 그 살인을 예방하는 데 그의 총을 파괴하는 것보다 더 좋은 방법이있을까? 그를 설득하는 것은 효과가 없을지도 모른다. 그를 죽이는 것은 그의 마음을 바꿀 가능성에 대한 억압이기도 하다. 같은 맥락에서 나는 압제자들과 그들을 돕는 이들의 도구를 파괴해야 한다. 그것은 전기 선인가, 가스탱크인가? 자물쇠, 철조망인가? 스턴건이나 순찰차인가?  

더 나아가, 그것은 엘리트들의 위장을 지속케 해주는 은밀한 기술인 걸까? 그것은 그들에게 값싸고 다스리기 쉬운 노동자와 소비자를 공급할 수 있는 효율적인 일부일처제 핵가족인가? 그것은 침묵하는 다수에 의해 현상유지되는 중산층의 안락한  무감각일까? 우리는 무엇을 방해해야 할까? 그것은 항상 달랐다. 도살장에 전기를 끊으려면 며칠의 계획과 1초의 수행 시간이 필요하지만, 가족과 사랑이 무엇인지에 대한 전통적인 관념의 힘을 끊으려면 평생이 걸린다. 

이런 사고방식조차 너무 낙관적인 것일까? 나는 이 글을 비교적 고립된 상태에서 쓴다. 나는 체제의 틈새에 대해 쓰지만, 내가 사랑하는 이들의 삶에도 틈새가 존재한다. 삶의 흡혈귀들로부터 시간을 빼앗기지 않는 한, 나는 그 틈새에 대해선 쓰지 않을 것이다. 친구들의 사랑에도 불구하고 나는 혼자일 것이라는 느낌을 떨칠 수 없다. 그리고 지금 내겐 거부할 수 있는 힘이 있는 반면, 다른 사람들은 그렇지 않다. 다른 사람들은 거부하길 원하지 않는다. 그들은 우릴 두들겨패던 아버지의 허리띠는 그의 것이 아니라, 그것을 비판하는 ``우리 자신의 것''이기도 하다고 말한다. 나는 이 주장을 나의 온 존재를 걸고 반박한다. 우리의 공격자에게 등에 9인치 깊이로 박힌 칼을 6인치만 빼라고 요구하는 것으로는 충분하지 않다는 것은 이미 언급되어 왔다. 필요하다면 우리는 어떤 수단을 써서라도 칼 전체를 꺼내야 한다. 다른 것이 있는가? 우리의 잠재의식은 우리 마음 속의 국가의 씨앗이 될 권력에 대한 지향과 다른가? 내가 사랑하는 사람들은 내가 혐오하고 파괴하고 싶은 사회와 타협한다. 내가 그들을 버려야 할까? 내가 기다리는 동안 시간이 흘러간다. 나는 그들이 편안한 상태에 의지하고, 심지어 중산층의 정서를 취하지 않을 것이라고 기대할 이유가 없다. 그럼에도 불구하고 나는 희망한다. 

나는 부족한 것이 있으면, 가책 없이 필요한 것을 취한다. 굶주림은 단순하다. 나는 한 끼를 해결할 수십 가지 방법을 알고 있다. 하지만 어떻게 하면 사람들로 하여금 편안함을 위해 기대, 자기 의심, 그리고 중독을 버리도록 재촉할 수 있을까? 어떻게 하면 그들이 나와 함께 이 사회에 전쟁을 선포하도록 재촉할 수 있을까? 어떻게 하면 무장의 기쁨을 사랑하도록 재촉할 수 있을까? 이것은 내가 훔치거나 거래하거나 구매할 수 없는 것이다. 지식인 부랑자는 태양을 향해 맹세한다! 나는 이곳이 싫다, 이 사회가 싫다. 매일같이 허무의 창조적인 공허를 찾는다! 그리고 새로운 해안으로 가는 고속도로. 하지만 내 동지들을 떠나기 위해? 사랑 가득한 일상을 떠나기 위해? 나는 그 생각을 증오한다. 나는 모든 것을 원하지만, 모든 것을 가질 가능성은 희박하다. 나는 기다린다.

그들은 왜 그것을 하는가? 그들은 내가 싫다고 한 것에 대해 미쳤다고 생각한다. 그러나 나는 오늘날의 그들과 똑같았다. 서류를 작성하고, 일정에 얽메여, 나의 구속이 내게 유용하다고 스스로 자위하고, 기대하는 이들의 반지에 입을 맞추었다. 놀라워, 앤디! 끝내줘, 한 잔 더! 그 터무니없는 과정을 한 번 더! 내년을 생각해봐! 다음 10년! 다음 인생! 이 천박한 위선이 내 영혼을 더럽혔다. 나는 나를 추스리고, 내 욕망의 고속도로에 올라탔다. 그리고 마음 내키는 대로 무엇이든지 했다. 안 되면 되게 했다. 나는 공범 없이 영원한 손님이 될 때까지, 다른 사람을 찾기 시작할 때까지, 허무한 일생으로부터 치유되기 시작할 때까지 위조하고, 훔치고, 일하고, 속이고, 표류하고, 연봉을 올렸다! 

안개가 자욱한 새로운 달 위, 우리 투쟁의 정점에서 반란군들은 소이탄을 설치할 만한 감옥의 약점을 주의 깊게 찾으면서 체제의 틈새에 살고 있다.

\section{}

예술가는 다른가?

우리는 재촉하거나, 해킹하거나 훔칠 수 없는 것을 유혹의 과정을 통해 조심스럽게 폐지해야 한다. 예술가는 자신이 들어갈 수 있는 죄수의 정신에서 따뜻한 틈새들(warm cracks)을 찾는다. 죄수는 감옥 벽의 사랑하는 목적을 자위적으로 강화한다는 거짓 가정 하에 예술가로 하여금 가까이 오도록 한다. 리듬은 최면을 걸고, 색은 눈을 망가뜨리고, 개념들은 스톡홀름 증후군을 자신의 욕구라 해설하고, 조만간 개인은 질문을 던지고 불신의 빛을 띠며 벽을 노려보고 있다. 죄수는 벽의 약점을 알아차리고, 다른 저항자에게서 온 쪽지를 본다. 사실 감옥 문이 잠겨 있지 않으며, 반대편의 동료 죄수들의 야유에도 불구하고 근처에 대기하고 있는 경비원이 없다는 것을 알게 된다. 

오히려 교도소 복도의 모퉁이에서 기다리는 것은 애송이 귀족이다. 탈주범들과 함께 전투에 뛰어드는 것은 그런 자유인들이다. 그러나 이러한 은유를 실생활로 옮기는데 있어서, 우리 각자는 이 탈출을 지리적, 문화적으로 위치시켜야 하는 과제를 안고 있다. 감옥을 나왔을 때, 우리는 무엇을 보는가?

\begin{quote}
시카고의 73세 노인 월터 운버하운(Walter Unbehaun)은 성인이 된 이후 일생의 대부분을 감옥에서 보냈다. 그는 잡힐 것을 알면서도 다른 은행을 털기로 결심했다. 법원서류들은 그가 감옥에서 더 편안함을 느꼈고, 여생을 그곳에서 보내기를 원했다고 주장한다.\parencite{atcl:ap2013}
\end{quote}

우리는 어떤 다른 것을 기대할 수 있을까? 동물 해방 운동가이자 상호교차성 페미니스트인 패트리스 존스(Pattrice Jones)는 에세이 <코끼리 짓밟기(Stomping with the Elephants)>\parencite{ibk:pattrice2006}에서 다음과 같이 말하고 있다. 

\begin{quote}
야생동물을 어떻게 길들일(break) 수 있는가? 열쇠는 단어 그 자체에서 찾을 수 있다. 당신은 연결을 끊는다. 

동물을 길들이기 위해서는 먼저 자연계에서 그 동물 각각을 물리적으로 격리시켜야 한다. 그런 다음 교미를 조절하고, 어미와 새끼의 관계를 끊고, 대가족의 구조를 파괴시킴으로써 다른 동물에 대한 모든 자연적 유대를 끊어야 한다. 동물을 자기 자신으로부터 멀어지게 하여, 더 이상 자신의 의지가 육체를 다스릴 것을 기대하지 않게 해야 한다. 마지막으로, 육체적, 성적 폭행 등 가능한 모든 방법으로 동물에게 굴욕감을 주고 폭력을 행사함으로써 그 정신을 깨뜨려야 한다. 

이런 것들이 아내를 통제하기 위해 학대하는 남편들이 사용하는 것과 같은 전술이거나, 야생 식물을 `재배'로 끌어들이기 위해 사용되는 비슷한 방법이라는 것은 우연이 아니다. 결국 `사육(husbandry)'은 농작물과 `가축'의 번식을 말하는 반면, `돌봄(groom)'은 말을 길들이는 조마사(the breakers of horses)이자 신부 들러리(the breakers of brides)를 말하게 되는 것이다.
\end{quote}

솔직히 말해도 될까? 우리 각자는 사육되어 온 것이다 우리에게는 천 년의 전통이나 지식이 없다. 우리는 권력이나 대상을 거치지 않은 어떤 사회적 또는 자연적 유대도 물려받지 못한다. 우리같이 가정적이지 않은 인간들 사이에는 빌어먹을 대가족적 공동체와 대등할 만한 유대가 존재하지 않는다. 우리는 학교, 성 역할, 직장, 가족관계, 우리가 영향을 주거나 창조하는 데 관여하지 않는 대중문화, 그리고 우리가 깊게 상호작용을 하거나 그 안에서 살지 않는 생태계에 대한 지배와 통제 체제에 의해 우리 자신의 개성으로부터 소외되어 왔다. 우리가 이런 붕괴로부터 홀로 벗어날 수 있다고 생각하는 것은 마르크스주의-교조주의적 개념인 `공산주의 혁명'이 일거에 일어난다는 것에 동의하는 것만큼이나 순진한 것이다. 우리의 자아의 혁명은 영속적이어야 하고 정적인 것과는 반대 방향에 자리 잡아야 한다. 

치유는 반드시 일어나야 하고, 타인과의 새로운 유대와 관계가 형성되어야 한다. 나의 사회에 대한 저항의 종점으로서 친교 집단을 추구하는 것은 아니다. 나는 감옥 밖에서 나 자신의 개성을 확고히 하기 위한 수단으로서, 생존 도구로서의 이런 사회적 모임을 추구한다. 나는 치유하기 위한 수단으로서 그리고 본질적인 수단으로서 그것을 찾고 있다.

그러나 이 치유는 엄밀히 말해서 재생되는 것은 아니다. 지배와 사육의 체제에 의해 우리에게 가해진 파괴는 윤곽이 뚜렷한 것이 아니다. 그것은 붕대를 감고 한 번에 치유할 수 있는 간단한 상처가 아니다. 그것은 오히려 너무 깊어서 우리의 몸과 마음, 그리고 세상의 재생 과정을 다시 거치는 장기간의 희생이 필요한 과정이다. 우리가 시작한 곳에서 끝나지 않을 정도로 치유하기 위해서는 이 체제가 우리의 ‘DNA’에 새긴 것을 표적으로 삼아 그것을 파괴해야 한다. 우리의 감옥으로 돌아가려는 경향은 우리 안에 깊이 새겨져 있다. 치유되기 위해서 우리는 이것을 파괴해야 한다. 

이와 같은 치유는 강요하거나, 도둑맞거나, 해킹 될 수 있는 것이 아니다. 우리로 하여금 우릴 속박한 것에 대해 의문을 품게 한 것과 동일한 동일한 유혹의 절차가 계속 되어야 한다. 사회에 대한 공격의 예술적 특성이 여기에 드러난다. 이런 관점에서 공격은 그 자체로 목적이 되어, 순수하게 효과적인 표현과 예술적 가치를 추구하게 된다. 우리의 행동이 우리의 무한한 힘을 일깨워주는 역할을 한다면, 그리고 그것이 우리의 자기 신념을 흔들고 우리의 친교 집단과 구성원을 강화시킨다면, '효용성'을 따지는 일은 무의미하다. 변증법은 죽었다. 그럼에도 불구하고, 우리의 행동이 개인에게 고통을 주는 가상의 감옥을 건설하는 객관화 장치에 맞서 싸워 부술 수 있다면, 더욱 더 좋은 일일 것이다. 

당신이 전속력으로 달리는 누군가의 어깨 너머로 화염과 불꽃, 복면의 폭도들이 장악한 거리를 쳐다보고, 이것들을 그저 당신의 돈으로 처리하려고 하는 건 우리의 주둔지를 속박 이전으로 되돌려놓는 불면의 자양분이다. 당신이 사표를 제출한 때의 느낌과 당신이 가장 마지막으로 즐기지 못할 일을 해야 했던 때가 언제인지 기억도 못한다는 점을 발견할 때의 느낌, 당신의 상사를 두들겨 팰 때의 느낌, 이것이 바로 귀족의 영혼을 살찌우는 기념비적인 순간들(pièce de résistance)이라고 할 수 있다. 숲이 돌아오는 것을 보고, 기름진 토양과 비옥한 정원을 가꾸고, 친구가 정신병에서 안정되는 것을 돕고, 친구들이나 동반자와의 오랜 사랑에 공을 들이고, 중독과 싸우는 것을 보는 것과 같은 장기적인 노력은 우리 반란군의 진지를 굳건히 하고 우리에게 성취감을 주는 데 기여한다. 물론 여기 내가 든 예시들은 각각의 자유로운 사람의 상상력에는 미치지 못할 것이다.

예술가이자 인습 타파주의자로서  `체포되지 않는' 자잘한 활동에 사로잡혀선 안된다. `자유는 모든 범죄를 포함하는 범죄다'라는 말이 있듯이, 권력이나 그 대상과의 타협을 거부하는 사람은 어떤 식으로든 국가와 대립할 수 밖에 없기 때문이다. 의심을 받는 것을 피하는 것은 딱 그 정도 효과만 가져올 뿐이다. 결국 최선의 노력에도 불구하고, 우리는 전방위 감시체제의 추적을 받을 것이다. 이 경우 어떻게 해야 하는가? 

\section{}

\begin{quote}
(전략) 유목민일수록 독립적이기 쉽다. 사막 유목민 문제를 해결하려는 광범위한 시도에서 볼 수 있듯이 정부는 이것을 알고 있다. 호주 원주민의 집요한 생존을 위한 생활 방식이든, 빅토리오(Victorio)가 이끄는 아파치의 비타협적 저항이든, 최근 사하라에서 일어난 투아레그 반란(Tuareg insurrection)이든, 유목민들은 대체로 싸움 및 도주에 능하다 (중략) 유목민들의 저항적 독립성이 종종 국경에 대한 실질적인 불신과 뒤섞인다는 것은 그들을 정부의 이념적 기반에 대한 위협이 되게 한다. -- \booktitlewrap{사막}
\end{quote}

역사적으로 고정된 지리적 거점을 고수하는 급진적 투쟁은 비교적 쉽게 억압되어 왔다. 파리 코뮌(The Paris Commune), 스페인 내전의 신디칼리스트(the Syndicalists), 멕시코의 마고니스타(The Magonistas), 마흐노브 자유 우크라이나(Mahknovist Free Ukraine) 등. 일부 예외는 이러한 주장의 `쉬운' 측면을 반증하는 것이지만, 어느 누구도 거점 기반 투쟁을 제한적인 의미에서 벗어나서 효과적인 반란 수단이라 제안할 수는 없다(나는 멕시코 남부에서 자파티스타(Zapatistas)의 상대적 성공은 다른 곳, 특히 산업화 이후의 서부에 도입될 수 없을 것 같은 특수한 사례라고 본다). 나의 견해로, 그것은 오히려 유목민적이고 널리 퍼진 방식으로 스스로 모여 가장 효과적으로 국가 권력과 탄압에 저항하는 사람들인 것처럼 보인다.

우리는 지금까지 상상해 왔던 그 자치 구역에 관한 모든 이상들을 버려야 한다. 도시 코뮌은 군과 경찰이 공격하기 쉬운 대상이다. 카탈루냐의 엘프렌테(El frente)는 종료되었고, 이전의 투쟁과 관계된 모든 집착들을 놓을 필요가 있다. 그 대신 우리는 우리의 귀족성을 완전히 잠재울 만큼 느슨하고 비공식적인 사회기반시설을 만들어야 한다. 19세기 미국의 노예화된 아프리카인들에게 자유를 가져다준 지하철과 같이, 우리는 국가 탄압을 한밤중의 두더지 잡기 게임으로 바꿀 수 있을 만큼 광범위하고 비공식적인 안전가옥, 자원 배급, 그리고 은밀한 의사소통의 네트워크를 만들어야 한다. 이런 생각은 전혀 새로운 것이 아니다. 1905년 12월 11일, 모스크바 반란군들에게 전송된 주의 사항은 다음과 같다. 

\begin{quote}
주요 규칙: 일제히 행동하지 마라. 최대한 서너 명씩 행동을 취하라. 가능한 한 많은 작은 집단이 있어야 하고 그들 각 집단은 빨리 공격하고 사라지는 법을 배워야 한다. 경찰은 백 명의 기마경찰 한 무리로 수천 명의 군중을 진압하려고 한다.

한 명을 물리치는 것 보다 백 명을 물리치는 것이 더 쉽다. 특히 그들을 기습한 후 비밀스럽게 사라질 수 있다면 더욱 그러하다. 이렇게 파악하기 힘든 작은 파견대들로 모스크바를 봉쇄한다면 경찰과 군대도 무력할 것이다. (중략) 거점들을 점령하지 마라. 군대는 언제든 그곳을 탈환할 수 있고 포병만으로 그곳을 파괴할 수 있다. 우리의 요새는 안마당이나 치고 나가기가 쉽고 떠나기 쉬운 곳이 될 것이다. 그들이 그것들을 점령한다고 해도 그들은 아무도 찾지 못할 것이고 많은 병사를 잃을 것이다. 그들이 그것들을 점령하는 것은 불가능할 것이다. 모든 집을 기동대로 채울 순 없을 테니까. 
\end{quote}

뉴욕 동남부에서 활동한 아나키스트 친교 단체, 후일 `검은 복면(Black Mask)'으로 이름을 바꾼 `궁지에 몰린 씨발것(Up Against the Wall Motherfucker)'은 `분석적인 거리 갱단'으로 알려졌다. 이것이 바로 우리의 노바토레 귀족들이 유목민적인 방식으로 간신히 조직되는 방식이다. 폭주족, 화물열차 무임승차 불량배(the crews of freight train hopping punks)의 일원, 그리고 전국 사막에서 버스에 야영하는 히피들의 지혜를 얻을 수 있는 곳이 여기다. `바이크로 치고 빠지기(strike and bike)' 사고방식은 우릴 예술가와 반란군으로 만들어줄 것이다. 특히 우리가 밤에 흙길과 좁은 길을 다닐 다용도 오프로드 바이크에 익숙해진다면 더욱 그러할 것이다. 

사회기반시설은 매우 중요하다, 그렇기 때문에 조심해야 한다. 지속적이든, 일회적이든 어떤 시도를 위해선  이미 만들어진 사회기반 시설이 필수적인 경우가 있다.

고전적 반란군들의 직접 행동이 목표를 공격하는 주요 수단이 될 필요가 없지만 보안의 관점에서 우리의 사회기반시설과 유목 생활에 이상의 방식으로 접근할 수 있다. 어떤 시도들은 완전한 익명성, 야간, 문명과의 합의를 필요로 하진 않는다. 종종 이런 요소들의 우리의 투쟁의 대부분을 차지하곤 하는데, 이를 위해선 체제 내 사회기반시설과 반 체제 내 사회기반시설 모두가 필요하다. 이러한 양상은 주로 고용을 회피하고 감옥 사회와 타협하기 위해 우리의 최저 생활 수단을 집산화하는 역할을 한다.

나는 `일하기 위해 삶을 팔고, 삶을 위해 노동을 지불해야 하는' 패러다임 밖의 생활을 위한 토지 점거에서 폐기물 재활용에 이르는 자원들에 대해 이미 언급했다. 여기에 작은 규모의 생계형 정원 가꾸기와 수렵도 추가될 수 있다. 외래종의 영역이 존재하며, (이러한 외래종들 중 가장 널리 분포하고 있는 교외와 도시에 있는 것들은 일반적으로 너무 창백하고 병들어서 맛이 없을지라도) 많은 것들은 식용과 보존용으로 적절하다. 영속 농업, 수렵, 소규모 정원 가꾸기에 대한 실제적인 고려 사항이 너무 많아서 이 에세이에서 모두 다루기는 힘들지만, 이러한 관행들은 경제의 가장자리에서 잘 사는 것에 대한 논의에서도 언급할 가치가 있으며, 유목민 사회기반시설의 구축을 위해선 반드시 고려되어야 한다. 특히 영속 농업은 북아프리카의 투아레그와 베르베르족에 의해 이미 어느 정도 개척된 `심고 내버려두고 되돌아오는' 방식으로 설정될 수 있다.

\begin{quote}
(아프리카 사하라의) 일부 유목민 집단은 때때로 마이다르(maidar) -- 수확을 위해 충분한 물을 보존하는 사막 표면의 저지대 -- 를 이용하여 곡물을 재배할 것이다. -- \booktitlewrap{식량과 농업}, 1991년 본문
\end{quote}

결국 가장 헌신적이고 잠을 싫어하는 예술가와 투사들도 쉬기 위해 머리를 뉘어야 한다. 집세를 피하려면 어디서 이렇게 할 수 있을까? 주(州)와 주 사이의 중간지대, 트럭 휴게소의 잔디밭, 도심 공원에 있는 나무의 해먹, 교외 쇼핑몰의 지붕에서 자는 것 등, 이런 것들은 이 질문에 대한 고전적인 해결책이다. 하지만 이 해결책은 저들이 없어지기 전까지만 사용할 수 있다. 경찰은 우리가 깨고 싶어 할 때보다 훨씬 이른 아침에 활동하는 폭력적인 부랑자들에게 질투심을 일으키는 방법을 알고 있다.

그 영역의 다른 쪽 끝에서, 우린 건물들을 무단 점거(squatting)할 수도 있다. (나는 바르셀로나에 무단 점거된 성이 있다는 말도 들었다!) 하지만, 괜찮은 피난처를 얻을 수 있는 이런 방법 역시, 특히 국가의 시야에 있는 사람들에게 문제가 없는 것은 아니다. 퇴거는 종종, 심지어 경고 없이 집행되기도 하며, 점거자들은 불법 침입 혐의로 체포될 수도 있다. 직접 행동에 대한 정보를 가지고 있는 사람들에겐 훨씬 더 중대한 혐의를 야기할 수 있다. 대체로 무단 점거는 기존의 ‘부랑자와 야영자’간의 충돌과 동일한 문제를 안고 있는 떠돌이들을 위한 미화된 형태의 야영지다.

자유 야영(free camping)과 무단 점거는 모두 특정한 상황에선 유용하지만, 나의 기본적인 생활에 유용한 것은 아니다. 내가 합리적인 `대안'으로 제안하는 것은 계절에 따라 야영을 할 목적으로 편의시설 없는 시골 땅을 임대하는 것이다. 나는 개인이 단순히 토지 소유자들에게 제안을 함으로써 필요하다면 `산간벽지' 에서 현금으로 매우 저렴하게, 1 에이커\footnote{에이커(Acre): 야드파운드법과 미국 단위계의 넓이의 단위이다. 1 에이커는 약 4,047 제곱미터에 해당하고, 이는 약 1,225 평에 대응한다.}의 땅을 임대할 수 있고 심지어 익명으로도 그것이 가능하다고 확신을 가지고 말할 수 있다. 일부 지역에서는 토지가 문제없이 무단 점거될 수도 있다(이것은 퇴거 시 아낌없이 떠날 수 있을 만큼 저렴한 잡동사니로 숙소를 건설함으로써 더 쉽게 이루어질 수 있다). 또한 중요한 것은 정부의 토지에서는 1 마일\footnote{마일(Mile): 야드파운드법과 미국 단위계의 길이단위이다. 약 1.6 킬로미터에 해당한다.} 씩 야영지를 옮기거나 2주마다 야영지를 옮기는 한 무기한으로 야영할 수 있다는 점이다.

전국의 어떤 필지든 날씨에 특별한 주의를 기울인다면 얼마든지 임차되거나 점거될 수 있다.  겨울에는 보다 남쪽으로 거처를 옮기는 식으로 말이다. 이동식 주거 체계로 티피(Tipis), 게르(Ger, Yurts), 집시의 벤더 천막(Bender tents), 천으로 만든 벽 텐트, 여행용 트레일러, 노점, 밴, 버스 등에서 선택해 볼 수 있다. 이상의 주거 체계들은 특정한 기후에 대응하는 장점을 가지고 있다. 또한 이들 대부분은 혹한과 혹서, 폭우와 같은 악조건에 대응해, 심지어는 호화롭게 살 수 있도록 설계될 수도 있다. 연구를 하라. 

기본적으로 지역 주민이나 임대인의 태도를 완전히 알 수 있을때 까진 살았던 흔적을 남기지 않는 편이 좋다. 그들과의  관계가 공고히 된 후에 장기적 거주를 위한 구조와 텃밭, 비축물 공급에 투자해야 할 것이다. 내가 당분간 가졌던 한 가지 아이디어는 사방에 차고의 문 형태의 문이 있는 둥글거나 아마도 8면의 구조물(이동식이든 아니든)일 것이다. 야영지를 지나는 각각의 거주자들은 이 문들 중 하나를 열고 그곳에 있는 천막을 갤 것이다. 이 구조물은 캠핑 트레일러만큼 정교한 특수한 미닫이 문을 설치해 바로 열고 들어올 수 있게 하거나, 훨씬 간단하게 그냥 천으로 가려두고 그 안에서 잘 수 있게 할 수도 있겠다. 이 구조의 장점은 공공설비와 독립된 샤워 시설, 주방, 소파, 심지어는 무료 도서관을 갖춘 공동 공간을 가질 수 있다는 것에 있다. 만약 겨울이 온다면 이 공간은 거주자들을 위해 효율적인 난방을 제공할 수도 있다.

십여 명의 개인들이 이러한 거주 구조를 위해 공동으로 각출할 수 있다. 이러한 비용을 10 내지는 12분의 1로 나누면 한 사람이 처리할 수 있을 정도로 충분히 낮다는 것을 고려하면, 그것은 전통적인 핵가족 단위로 임대료를 지불하는 것보다 훨씬 저렴한 생활 수단임이 입증될 것이다. 이 공간들은 또한 대량의 보존 식품, 비축 식량, 돈, 생필품을 위한 협동조합의 중추 역할을 할 수도 있다. 이 시점에 이르면 텃밭과 영속 농업 사업이 시작될 수 있다. 게다가, 육아와 기술 공유와 같은 비물질적인 형태의 상호 부조가 이루어질 수 있다. 

이런 맥락에서 임대는 장점이 많다. 거래는 현금으로, 그리고 심지어 (필요하다면) 익명으로 그리고 은밀히 이루어질 수 있기 때문에, 당신의 존재가 의혹을 불러일으킬 가능성이 거의 없을 것이다. 지방 당국도 알 수 없을 것이고, 당신의 토지 소유자는 당신이 하고 있는 일에 대해 거의 신경을 쓰지 않을 수도 있다. 물론 자신이 원하는 곳으로 자유롭게 이주하길 원하는 유목민은 임대인에 까다로울 수 있으며, 오로지 게으르고 무관심한 유형의 임대인에게만 임대받길 원할 수도 있다. 어떤 경우에는 임대인이 당신의 캠프를 거의 볼 수 없을지도 모른다. 그리고 일반적으로 개발되지 않은 비농업용 토지는 상당히 낮은 비용으로 임차할 수 있다. 어떤 경우에는 이 땅을 사용하기 위한 노동 교류(work-trade)\footnote{숙식을 제공받는 대가로 일정한 노동력을 제공하는 형태의 여행 형태이다. Workway(workaway.info)나 World Packers(www.worldpackers.com)와 같은 서비스가 이런 형태의 교류를 주선하고 있다.}도 생각해볼 수 있다. 마지막으로 당신과 토지 사이에 어떠한 법적 관계가 없기 때문에, 필요하다면 흔적도 없이 계약을 쉽게 종료할 수도 있다. 

우리는 이 전초기지로부터 자유롭게 공격할 수 있다. 사막과 숲의 이 은밀한 헛간들이 자유주의자, 파시스트, 외로운 혁명가들에 의해 요리되는 사회 질서, 뭐가 되었든 내일의 질서에 대한 증오로 들썩이게 하라. 우리의 화염은 불확실성을 가속시킬 것이고, 권태를 해소할 것이다! 우리는 모든 왕좌를 썩게 하기 위해 이곳에 왔다. 우린 생명의 강탈자와 환영의 카르텔의 기념비를 녹슬게 하기 위해 이곳에 왔다. 우리는 신월을 타고 이곳에 내려와 모든 감옥의 벽을 부술 것이다. 우린 결코 막을 수 없는 불빛이 될 것이다! 다시 한 번 우리는 분명히 원한다. ``이 비극적인 사회의 황혼이 우리에게 `나 자신'을, 약간의 안식을, 황홀한 우주적 빛에 이르는 불꽃을 주기를 원한다!''\parencite{ibk:renzo2012}

\section{}

우리는 황충처럼 우리의 전술을 미워하는 이들에게 떼거리로 달려들어 독액으로 그들을 흠뻑 적셔 끝장내거나 분해시켜버릴 것이다. 다소 흠결이 있지만, 미래주의자들이 말했듯 ``어떤 사람들은 늙은 채로 태어난다. 과거의 망령들은 침을 질질 흘리고, 암호문은 독소에 퉁퉁 부웠다. 그들에겐 말이나 생각이 아니라 단 하나의 금지령만이 유효하다: \textbf{종료.}''\parencite{atcl:francesco1910} 여기 무고한 이주민들을 두들겨패는 파시스트들을 쏴버릴 당신을 위한 권총이 있다! 우리에겐 이 행성을 난도질하고 유독가스로 뒤덮는 당신들의 살상기계에 대해 심장으로부터 터져나오는 파괴 명령을 내린다! 우린 젊은이들을 속여 사악한 타협안을 주입시키는 진보적인 머저리들의 정체된 전술과 지루한 담론을 폭로하고 중지시킬 것이다! 무덤에서 필리피의 유령이 선언한다. \textit{``세상은 무너져내리고, 사방에서 피와 시체와 썩어가는 것을 보니 이 얼마나 기쁜 일인가!''\parencite{ibk:bruno2008}}

나는 이제 숨어다니며 가련한 희망을 과시하는 것엔 아주 넌더리가 난다. 나는 변증법이라는 집요한 유령 때문에 병들었다. 진심으로 나는 나를 둘러싸고 있는 공허를 끌어안는다. 진심으로 나는 공명하는 자아 속으로, 그 깊은 애정 속으로 뛰어든다! 내 동지들, 절대 거부로 윤기가 흐르는 악당들이여, 그대들과 나는 어제의 거뭇한 고속도로에서 벗어나 스모그의 해맞이 안으로 간다! 당신과 함께 나는 자유 속에서, 그리고 파멸 속에서 죽을 것이다! 그때까지 삶, 틈새의 삶으로! 

\begin{quote}
당신이 어떤 사회를 건설하든 한계를 가질 것이다. 그리고 영웅적인 부랑자들은 그 사회가 무엇이든 한계 밖에서 무질서하고 거칠고 순수한 생각으로 활보할 것이다. 그들은 새롭고 무시무시한 반란 모의 없이는 살 수 없는 자들이다

나도 그들 중 하나일 것이다! 

내 앞의, 내 뒤의 동료들에게 말한다. ``그러니 신이나 너의 우상이 아닌 네 자신에게 돌아가라. 네 안에 숨어 있는 것을 찾아라. 그것에 빛을 비춰라. 네 자신을 보여라!'' 

내밀하게 자신의 내면을 탐구해온 모든 사람들은 태양 아래 존재할 수 있는 어떤 형태의 조직과 사회도 가릴 수 있는 그림자니까! 부랑자들, 반골들, 별종들, 신념의 지배자들, 그리고 무용의 정복자들과 같은 경멸스런 귀족들이 단호히 전진할 때 모든 사회는 공포에 떨 것이다. 

그러니, 인습 타파주의자들이여 가자, 앞으로! 

\textit{\textbf{``이미 하늘은 불길하고 암흑에 휩싸여 조용해질 것이다!''}}\parencite{ibk:renzo2012}
\end{quote}
\end{article}

\chapter{편집자 후기}

<<틈새의 삶>>은 2015년 경 공개되어 현재까지 영미권의 아나키스트들이 자주 인용하는 에세이이다. 이 선언문의 저자인 프리드리히 루럴 루시퍼는 이 글을 마지막으로 해당 이름을 사용하지 않는 것으로 보인다. 

이 에세이에 언급된 몇 가지 낯선 키워드와 이름, 제목에는 영어 표현을 함께 표기했다. 우리로 하여금 한국어로 상상해본 적 없는 10년대 허무주의적 아나키즘의 지평과 그 너머를 가늠할 수 있게 해줄 것이다. 부지런한 이들이라면 저 단어들을 단서 삼아 이미 그곳으로 출발할지도 모르겠다. 참조 문헌 목록은 영어로 가득하지만 젊은 아나키즘 실천가들은 상황을 언제고 이따위로 내버려두진 않을 것이라 믿는다. `나도 그들 중 하나일 것이다.'

이 에세이의 제안들이 남한의 상황과는 다소 맞지 않아 보일 수 있다. 하지만 결국 이 제안들은 `체계의 약점'과 `소이탄을 설치할 만한 감옥의 약점', `지배적인 문화와 그 부역자들이 보거나 만질 수 없는 공간'을 찾는 지혜의 단서라고 할 수 있다. 이 점에 비춰보면 남한의 `부랑자들, 반골들, 별종들, 신념의 지배자들, 그리고 무용의 정복자들'의 급진적인 실천을 위해 틈새를 찾는 작업에 시사하는 바가 있으리라 믿는다. 분명 남한 어딘가에는 장난감 총으로 반란을 모의하는 것보다 더 나은 급진적 실천이 존재할 것이다. 이 책으로 그들에게 기여할 수 있길 바란다.

\begin{flushright}
2019년 11월, 김미루
\end{flushright}

\backmatter

\printbibliography[title={참조문헌}]

\begin{lastnote}
\begin{description}[itemsep=1pt,parsep=1pt]%
\item[제목] \fulltitle
\item[저자] \theauthor
\item[번역] \translator{}
\item[검수] 전규리, 에이미 앰플
\item[편집] 김미루
\item[발행] 흑묘단, 금치산자레시피
\item[디자인] 써드엔지니어링카르텔
\item[출간일] \thedate
\end{description}

\begin{description}[itemsep=1pt,parsep=1pt]%
\item[출판] 금치산자레시피
\item[이메일] gtsz.rcp@gmail.com
\item[웹사이트] http://gtszrcp.com
\item[인스타그램] gtsz.rcp
\end{description}

\begin{description}[itemsep=1pt,parsep=1pt]%
\item[저작권]%
이 책은 The Anarchist Library를 통해 Anti-Copyright 저작권 선언과 함께 공개된 <<Life in the Cracks ---  A Novatorean Essay on Creating Meaning From Nothing on a Hopelessly Dying Planet >>(https://theanarchistlibrary.org/library/friedrich-rural-lucifer-life-in-the-cracks)를 번역한 책입니다. 이 책의 저작권은 크리에이티브커먼즈 저작자표시-동일조건변경허락 4.0 국제 라이센스에 의해 보호됩니다.
\end{description}
\end{lastnote}
\end{document}